\documentclass[16pt]{article}

%\usepackage{amssymb}
\usepackage{indentfirst}
\usepackage{algorithm}
\usepackage{algorithmic}
\usepackage{listings}
%\lstset{language=c, frame=single, rulesepcolor=\color{blue}}

\lstset{ %
language=c,                % the language of the code
basicstyle=\footnotesize,       % the size of the fonts that are used for the code
%numbers=left,                   % where to put the line-numbers
%numberstyle=\footnotesize,      % the size of the fonts that are used for the line-numbers
%stepnumber=2,                   % the step between two line-numbers. If it's 1, each line
                                % will be numbered
%numbersep=5pt,                  % how far the line-numbers are from the code
backgroundcolor=\color{white},  % choose the background color. You must add \usepackage{color}
showspaces=false,               % show spaces adding particular underscores
showstringspaces=false,         % underline spaces within strings
showtabs=false,                 % show tabs within strings adding particular underscores
frame=single,                   % adds a frame around the code
tabsize=2,                      % sets default tabsize to 2 spaces
captionpos=b,                   % sets the caption-position to bottom
breaklines=true,                % sets automatic line breaking
breakatwhitespace=false,        % sets if automatic breaks should only happen at whitespace
title=\lstname,                 % show the filename of files included with \lstinputlisting;
                                % also try caption instead of title
%escapeinside={\%*}{*)},         % if you want to add a comment within your code
%morekeywords={*,...}            % if you want to add more keywords to the set
}

%\usepackage{mdwtab}
\usepackage{graphicx}
\usepackage{hyperref}
\usepackage{color}
\usepackage{fullpage} %use this package to the standard 1 inch margins on all sides
\begin{document}
\title{\huge \textbf{Wayne State University \\
 \vline \\
\Huge CSC 4421 - Spring/Summer 2017 \\
 Computer Operating Systems Labs\\
 Lab 3 - Process control}\\
 \vline
 }
\author{\textbf{Instructor}  \vspace{0.2cm} \\ Saeid Balaneshin-kordan\\
}
\date{}
\maketitle

\begin{center}
\Large Points Possible: 100

\textcolor{red}{Due: June 6, 2016}
\end{center}

\section*{Goals}

The purpose of this lab is to help you learn about fork() and wait() system calls in C.

\section*{Tasks}

\subsection*{Task 1}

\begin{enumerate}
\item define a function to carry out the task corresponds to a child process within the operating system as:
\begin{enumerate}
\item this function should use a system call to get its corresponding process identification.
\item this function should print the obtained process ID on the console.
\item this function prints either `a' or `b'. Whether it should print `a' or `b' should be determined by one of the arguments of this function.
\item this function should at the end use a library call that cause normal process termination. 
\end{enumerate}

\item the main function create two separate child processes as:
\begin{enumerate}
\item the main function should call a system call twice to create two separate child processes
\item by using if statements, the main function should call the function that was defined for the child processes by given this function appropriate inputs. This function should be called twice for each of the child processes.
\item by using a system call, this function should wait for both of its child processes to terminate.
\item print an end of line character at the end
\end{enumerate}

\item run your program and observe its output. It should print out either "ab" or "ba" depending on whether the first or second child processes are invoked first.

\item write a bash script that call this program for 100 times and counts the times it prints "ab".

\item add an extra wait system call to the code, so that the code prints only "ab".  


\end{enumerate}
\end{document}



\end{document} 

