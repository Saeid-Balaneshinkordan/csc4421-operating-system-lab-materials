\documentclass[16pt]{article}

%\usepackage{amssymb}
\usepackage{indentfirst}
\usepackage{algorithm}
\usepackage{algorithmic}
\usepackage{listings}
%\lstset{language=c, frame=single, rulesepcolor=\color{blue}}

\lstset{ %
language=c,                % the language of the code
basicstyle=\footnotesize,       % the size of the fonts that are used for the code
%numbers=left,                   % where to put the line-numbers
%numberstyle=\footnotesize,      % the size of the fonts that are used for the line-numbers
%stepnumber=2,                   % the step between two line-numbers. If it's 1, each line
                                % will be numbered
%numbersep=5pt,                  % how far the line-numbers are from the code
backgroundcolor=\color{white},  % choose the background color. You must add \usepackage{color}
showspaces=false,               % show spaces adding particular underscores
showstringspaces=false,         % underline spaces within strings
showtabs=false,                 % show tabs within strings adding particular underscores
frame=single,                   % adds a frame around the code
tabsize=2,                      % sets default tabsize to 2 spaces
captionpos=b,                   % sets the caption-position to bottom
breaklines=true,                % sets automatic line breaking
breakatwhitespace=false,        % sets if automatic breaks should only happen at whitespace
title=\lstname,                 % show the filename of files included with \lstinputlisting;
                                % also try caption instead of title
%escapeinside={\%*}{*)},         % if you want to add a comment within your code
%morekeywords={*,...}            % if you want to add more keywords to the set
}

%\usepackage{mdwtab}
\usepackage{graphicx}
\usepackage{hyperref}
\usepackage{color}
\usepackage{fullpage} %use this package to the standard 1 inch margins on all sides
\begin{document}
\title{\huge \textbf{Wayne State University \\
 \vline \\
\Huge CSC 4421 \\
 Computer Operating Systems Labs\\
 Lab 2 - System Calls}\\
 \vline
 }
\author{\textbf{Instructors}  \vspace{0.2cm} \\
Saeid Balaneshin
}
\date{}
\maketitle

\begin{center}
\Large Points Possible: 100

\end{center}

\section*{Goal}

To get familiar with Directory Operations.

\section*{Tasks}

\begin{enumerate}
\item How to use ``find'' command to find dirent.h in /usr? This header file defines an struct for format of directory entries called ``dirent''.
\item In your code, define a pointer to the struct ``dirent''.
\item dirent.h defines a struct called DIR. What is definition of DIR according to dirent.h?
\item dirent.h defines a function called opendir. What is definition of opendir according to dirent.h?
\item What is the description of opendir according to ``man''?
\item Include sys/type.h as a header file in your code.  
\item How to use ``find'' command to find sys/type.h in /usr? 
\item What sys/type.h contains?
\item Define a pointer with type directory stream (\texttt{DIR}).
\item The program needs to get an argument for the directory name. Check if the number of arguments (argc) is two, otherwise prints an error and exits the program.
\item Your program needs to open the directory given by
argument \texttt{argv[1]}. 
 Store it in your defined directory stream.
 Also check if the directory name is opened correctly,
 otherwise quit the program with an error message.
 \item Your program should read all the entries in this directory, and
then print all the file names and their sizes on the console. You may need to use ``stat'' function.
\item Your program should close the defined directory stream and check if the file is closed correctly.
\end{enumerate}

\end{document}



\end{document} 

