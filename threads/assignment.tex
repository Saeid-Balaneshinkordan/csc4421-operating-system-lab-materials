\documentclass[16pt]{article}

%\usepackage{amssymb}
\usepackage{indentfirst}
\usepackage{algorithm}
\usepackage{algorithmic}
\usepackage{listings}
%\lstset{language=c, frame=single, rulesepcolor=\color{blue}}

\lstset{ %
language=c,                % the language of the code
basicstyle=\footnotesize,       % the size of the fonts that are used for the code
%numbers=left,                   % where to put the line-numbers
%numberstyle=\footnotesize,      % the size of the fonts that are used for the line-numbers
%stepnumber=2,                   % the step between two line-numbers. If it's 1, each line
                                % will be numbered
%numbersep=5pt,                  % how far the line-numbers are from the code
backgroundcolor=\color{white},  % choose the background color. You must add \usepackage{color}
showspaces=false,               % show spaces adding particular underscores
showstringspaces=false,         % underline spaces within strings
showtabs=false,                 % show tabs within strings adding particular underscores
frame=single,                   % adds a frame around the code
tabsize=2,                      % sets default tabsize to 2 spaces
captionpos=b,                   % sets the caption-position to bottom
breaklines=true,                % sets automatic line breaking
breakatwhitespace=false,        % sets if automatic breaks should only happen at whitespace
title=\lstname,                 % show the filename of files included with \lstinputlisting;
                                % also try caption instead of title
%escapeinside={\%*}{*)},         % if you want to add a comment within your code
%morekeywords={*,...}            % if you want to add more keywords to the set
}

%\usepackage{mdwtab}
\usepackage{graphicx}
\usepackage{hyperref}
\usepackage{color}
\usepackage{fullpage} %use this package to the standard 1 inch margins on all sides
\begin{document}
\title{\huge \textbf{Wayne State University \\
 \vline \\
\Huge CSC 4421 - Spring/Summer 2017 \\
 Computer Operating Systems Labs\\
 Lab 4 - Threads}\\
 \vline
 }
\author{\textbf{Instructor}  \vspace{0.2cm} \\ Saeid Balaneshin-kordan\\
}
\date{}
\maketitle

\begin{center}
\Large Points Possible: 100

\textcolor{red}{Due: June 11, 2017}
\end{center}

\section*{Goals}

The purpose of this lab is to help you learn about POSIX Threads programming.

\section*{Tasks}

\subsection*{Task 1}
In this task, we create and deal with multiple threads.
\begin{enumerate}
\item Write a program that gets an integer named ``numberOfThreads'' as its argument
\item Use a for loop and create ``numberOfThreads'' new threads in the calling process.
\item Define a function named ``worker'' that its return type and argument type are void* such that:
\begin{itemize}
\item It gets an integer named ``threadNumber'';
\item It prints ``threadNumber'' on the console;
\item It uses a nested for loop to impose a high CPU load for a couple of seconds;
\item It finds the square of ``threadNumber'' and stores in a global int array (say ``squareNumbers'') that keeps this value for all the threads;
\item It prints a message that it is going to terminate the thread;
\item terminates calling thread;
\item This function should be invoked in the mentioned loop that creates new threads.
\end{itemize}
\item Use a for loop and that waits for all the created threads to terminate.
\item Print all the items in the array ``squareNumbers''
\item Run your program and also ``top'' (or ``htop'') at the same time and take a screen shot of your program's CPU usage.
\end{enumerate}
\end{document}



\end{document} 

