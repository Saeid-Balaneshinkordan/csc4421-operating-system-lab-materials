\documentclass[16pt]{article}

%\usepackage{amssymb}
\usepackage{indentfirst}
\usepackage{algorithm}
\usepackage{algorithmic}
\usepackage{listings}
%\lstset{language=c, frame=single, rulesepcolor=\color{blue}}

\lstset{ %
language=c,                % the language of the code
basicstyle=\footnotesize,       % the size of the fonts that are used for the code
%numbers=left,                   % where to put the line-numbers
%numberstyle=\footnotesize,      % the size of the fonts that are used for the line-numbers
%stepnumber=2,                   % the step between two line-numbers. If it's 1, each line
                                % will be numbered
%numbersep=5pt,                  % how far the line-numbers are from the code
backgroundcolor=\color{white},  % choose the background color. You must add \usepackage{color}
showspaces=false,               % show spaces adding particular underscores
showstringspaces=false,         % underline spaces within strings
showtabs=false,                 % show tabs within strings adding particular underscores
frame=single,                   % adds a frame around the code
tabsize=2,                      % sets default tabsize to 2 spaces
captionpos=b,                   % sets the caption-position to bottom
breaklines=true,                % sets automatic line breaking
breakatwhitespace=false,        % sets if automatic breaks should only happen at whitespace
title=\lstname,                 % show the filename of files included with \lstinputlisting;
                                % also try caption instead of title
%escapeinside={\%*}{*)},         % if you want to add a comment within your code
%morekeywords={*,...}            % if you want to add more keywords to the set
}

%\usepackage{mdwtab}
\usepackage{graphicx}
\usepackage{hyperref}
\usepackage{color}
\usepackage{fullpage} %use this package to the standard 1 inch margins on all sides
\begin{document}
\title{\huge \textbf{Wayne State University \\
 \vline \\
\Huge CSC 4421 - Spring/Summer 2017 \\
 Computer Operating Systems Labs\\
 Lab 7 - Process Scheduling}\\
 \vline
 }
\author{\textbf{Instructor}  \vspace{0.2cm} \\ Saeid Balaneshin-kordan\\
}
\date{ }
\maketitle

\begin{center}
\Large Points Possible: 100

\textcolor{red}{Due: July 2, 2017}
\end{center}

\section*{Goals}

The purpose of this lab is to help you learn about Process Scheduling.

Every red star represents something you need to submit. Add each item with a red start to a .docx file, and submit just that file to blackboard. When it asks for your file, you can just include the contents of the file. No zipped folder is required.

\section*{Tasks}

\subsection*{Task 1$^*$ (50 points)}
\textbf{Process Simulator}
In this task, you are required to run a process simulator and put your answers to the following questions in a file Task\_1\_lastname\_firstname.docx. Use one file with all of the answers and screen-shots. Upload the file to Blackboard(no need to zip).

\begin{enumerate}
\item Download process scheduler from the following web site.

\url{http://vip.cs.utsa.edu/simulators/}

\item Use ``sudo apt-get install'' in the command line to install c shell and java. You should use install ``default-jre'' and ``csh". If you don't know how to install java in linux, please google it. Or you can use Windows.

\item Type ``csh'' to change into c shell mode. Start the simulator by run ``./runps'' or in Windows system, you should click runps.bat file.

\item  {\it By default, the simulator will run two experiments. The first experiment uses FCFS
scheduling algorithm, and the second one uses SJF. The program creates lots of logs. Some information
is shown as soon as the experiment is done. Some other inforamtion can be accessed from various other buttons on the gui, such as ``All(\#)'' which shows process information, and ``Show All Table Data'' which shows some information on each of the runs.} Click the `Run Experiment' button in the
simulator. Explore the simulator and answer the following questions:

\begin{enumerate}
\item \textcolor{red}{$^{*}$} How many events were created for each experiment?
\item \textcolor{red}{$^{*}$} How many milliseconds did each take?
\item \textcolor{red}{$^{*}$} In both experiments, how many process are created?

\end{enumerate}

\item Restart the simulator again, this time you will only run the FCFS algorithm (by clicking
Run so that it changes from ``Run All'' to ``Run: myrun 1").

\begin{enumerate}
\item Look at the process information by clicking the `All(\#)' button in the simulator.
\item Look at the CPU history information by clicking the `CPU History(\#)' button in the simulator.
\item \textcolor{red}{$^{*}$} Explain how process 30 is scheduled.
\end{enumerate}

\item Restart the simulator again, this time you will only run the SJF algorithm (by clicking
Run so that it changes from ``Run: myrun 1'' to ``Run: myrun 2").

\begin{enumerate}
\item Look at the process information by clicking the `All(\#)' button in the simulator.
\item Look at the CPU history information by clicking the `CPU History(\#)' button in the simulator.
\item \textcolor{red}{$^{*}$} Explain how process 30 is scheduled.
\item \textcolor{red}{$^{*}$} Explain why process 30 is scheduled differently compared to the process 30 in the last experiment.
\end{enumerate}

\item Restart the simulator again. Run both scheduling algorithms at once (Run All).
Draw the following figures (by clicking on the ``Graph Type'' button until it shows
the appropriate graph, and then clicking the ``Draw'' button.)

\begin{enumerate}
\item \textcolor{red}{$^{*}$} Draw the waiting time, waiting time box and waiting ratio figures. Include the figures in your file.
\begin{enumerate}
\item \textcolor{red}{$^{*}$} Which algorithm has shorter waiting time? Explain the reason.
\item \textcolor{red}{$^{*}$} Why is the maximum waiting time for SJF larger than FCFS?
\item \textcolor{red}{$^{*}$} Give some comments about FCFS's and SJF's waiting time in your own
words. i.e. Pros and Cons.
\end{enumerate}

\item \textcolor{red}{$^{*}$} Draw `turnaround' and `turnaround box' figures. Include the figures in your file.
\begin{enumerate}
\item \textcolor{red}{$^{*}$} What does turnaround time mean?
\item \textcolor{red}{$^{*}$} From the figures, which algorithm has lower minimum turnaround time? Explain the reason.
\end{enumerate}

\item \textcolor{red}{$^{*}$} Draw `Gantt Chart' for both FCFS and SJF. Include the figures in your file.
\end{enumerate}
\end{enumerate}

\newpage
\subsection*{Task 2}

Read the following guide about the process scheduling simulator. Learn how to change simulation settings.

\subsubsection*{Specifying an Experiment}

An experiment is specified by two files. Each file has a name consisting of a base name
and an extension. The files are:

\begin{itemize}
\item The experiment specifies a number of experimental runs.
Each run is specified by a base name and a possible list of modifications.

\item The experimental run file, which is also called the run file for short contains information about the scheduling algorithm to use,
the processes to run, and when the processes arrive.
\end{itemize}

\subsubsection*{Time}

The simulator uses a virtual time which is represented by a floating point value of
unspecified units. When the simulator is started, the time is set to 0.0. The simulator is
event driven, and events that occur at the same time may occur in any order.

\subsubsection*{Processes}

A process is specified by the following information:

\begin{itemize}
\item Arrival time
\item Total CPU time
\item CPU burst time distribution
\item I/O burst time distribution
\end{itemize}

\subsubsection*{Probability Distribution}

The distribution supported by the simulator are

\begin{itemize}
\item Constant distribution. Key word: constant
\item Exponential distribution. Key word: exponential
\item Uniform distribution. Key word: uniform
\end{itemize}

\subsubsection*{The structure of a run file}

An experimental run contains all of the information needed to run the simulator on one
collection of processes. An experimental run specifies a scheduling algorithm, a
collection of processes, and when the processes arrive. A simple format for an
experimental run file is like :

\begin{tabular}{l l}
\textbf{name} & //experimental run name \\ 
\textbf{comment} & //experimental run description \\
\textbf{algorithm} & //algorithm description \\
\textbf{numprocs} & //number of processes \\
\textbf{firstarrival} & //first arrival time \\
\textbf{interarrival} & //inter-arrival distribution \\
\textbf{duration} & //duration distribution \\
\textbf{cpuburst} & //cpu burst distribution \\
\textbf{ioburst}  & //io burst distribution \\
\textbf{basepriority} & //base priority 
\end{tabular}


Here is a sample experimental run file that must be stored in the file \textbf{myrun.run}.

\lstinputlisting{myrun.txt}

This file specifies a run of 20 processes using the shortest job first algorithm. The first
process arrives at time 0.0. The inter-arrival times are all 0.0, so all processes arrive at the
same time. Each process has a duration (total CPU time) chosen from a uniform
distribution on the interval from 500 to 1000. All processes have a constant cpu burst
time of 50 and a small constant I/O burst time of 1.0. The base priority must be present in
the file, but it is currently not used by the simulator.

\subsubsection*{The structure of an exp file}

An experiment specifies a number of experimental runs that are to be made, compared
and analyzed. There are an arbitrary number of lines specifying experimental runs to be
made.

The simulator allows you to do this by specifying the same experimental run in each case
and giving a new value to one or more parameters. The general format for a run line in
the experiment file consists of the word \textbf{run} followed by the name of an experimental
run, followed by a list of modifications to that experimental run. So it looks like this

\begin{tabular}{l l l}
name  & experiment\_name & \\
comment & experiment\_description & \\
run & run\_name1 & optional\_modification\_list1 \\
... & &\\
run & run\_namen & optional\_modification\_listn \\
\end{tabular}

Here is an example experiment file that must be stored in the file \textbf{myexp.exp}

\lstinputlisting{myexp.txt}

This experiment file makes three runs. All of the runs are based on the run file
\textbf{myrun.run} above. In the second run the CPU burst distribution is changed to be uniform
in the interval from 10 to 90. In the third run the CPU burst is an exponential distribution
with mean 50.
Here is a list of the key words you can modify:


\begin{itemize}
\item \textbf{numprocs}: the number of processes to create.
\item \textbf{firstarrival}: the arrival time of the first process, a floating point number.
\item \textbf{basepriority}: the base priority of the processes created. This base priority is not
used unless the simulator is using priorities.
\item \textbf{interarrival}: the distribution of the interarrival times of the processes.
\item \textbf{duration}: the distribution of the total CPU time used by the processes.
\item \textbf{cpuburst}: the distribution of the CPU burst times of the processes.
\item \textbf{ioburst}: the distribution of the I/O burst times of the processes.
\item \textbf{algorithm}: the scheduling algorithm to use.
\end{itemize}


\subsubsection*{Scheduling Algorithms Supported}

\begin{itemize}
\item  Round Robin (RR):
\item  First-Come/First-Served (FCFS): Processes are taken from the ready queue in
   the order that they arrived. Once a process is using the CPU, it stays there until its
  CPU burst expires. It then goes into the I/O waiting queue until its I/O burst
 expires.
\item  Shortest Job First (SJF): The next process to be removed from the ready queue
   is the one with the shortest CPU burst time. If there is more than one process with
  the smallest time, the one that arrived first is taken.
\end{itemize}
A scheduling algorithm is specified with a string and a possible optional floating point
parameter depending on the scheduling algorithm. The following strings are used to
specify scheduling algorithms:

\begin{itemize}
\item \textbf{RR} quantum represents the Round Robin algorithm with the given quantum.
\item \textbf{FCFS} represents the First-Come/First-Served algorithm.
\item \textbf{SJF} represents Shortest Job First scheduling.
\end{itemize}

\newpage
\subsection*{Task 3\textcolor{red}{$^{*}$} (25 points): FCFS and SJF}

\begin{itemize}
\item Modify the \texttt{myexp.exp} file to have 2 run lines that will use the following
algorithms:

FCFS, SJF, On each run line, make sure that both the algorithm and key are
appropriately set.
\item Now modify the myrun.run file so that there will be a total of 20 processes with
the following properties:

\begin{itemize}
\item Do not change the seed line of the run file.
\item All processes arrive at time 0.0.
\item All processes have a constant duration of 100.
\item All have constant I/O bursts of 10.
\item All have basepriority 1.0.
\item The first 10 processes have CPU bursts uniformly distributed between 2
and 8.
\item The last 10 processes have CPU bursts uniformly distributed between 50
and 60.
\end{itemize}
\item Run the simulator using FCFS and SJF and draw a `Gantt chart' for each of the two ().
Add these two figures along with the revised myexp.exp and myrun.run files to your file.
\item Did you find any difference from those two figures? Briefly explain the impact of
these two scheduling algorithms?

\end{itemize}

\subsection*{Task 4\textcolor{red}{$^{*}$} (25 points): Round Robin}

\begin{itemize}
\item Modify the \texttt{myexp.exp} file to have 4 run lines that will use the following
algorithms:

RR 5, RR 20, RR 100, and RR 200. On each run line, make sure that both the
algorithm and key are appropriately set like this:

run myrun algorithm RR 200 key ``RR 200"

\item Now modify the myrun.run file so that there will be a total of 20 processes with
the following properties:

\begin{itemize}
\item Do not change the seed line of the run file.
\item All processes arrive at time 0.0.
\item All processes have a constant duration of 1000.
\item All processes have constant CPU bursts of 101.
\item All have constant I/O bursts of 100.
\end{itemize}

\item Run the simulator using the revised experiment setting. Draw the `Gantt chart' for each run. Submit these 4 figures along with the revised myexp.exp and myrun.run files.

\item Briefly describe the Round Robin algorithm, and the impact of using different quantums.

\end{itemize}
\end{document}
