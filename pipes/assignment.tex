\documentclass[16pt]{article}

%\usepackage{amssymb}
\usepackage{indentfirst}
\usepackage{algorithm}
\usepackage{algorithmic}
\usepackage{listings}
%\lstset{language=c, frame=single, rulesepcolor=\color{blue}}

\lstset{ %
language=c,                % the language of the code
basicstyle=\footnotesize,       % the size of the fonts that are used for the code
%numbers=left,                   % where to put the line-numbers
%numberstyle=\footnotesize,      % the size of the fonts that are used for the line-numbers
%stepnumber=2,                   % the step between two line-numbers. If it's 1, each line
                                % will be numbered
%numbersep=5pt,                  % how far the line-numbers are from the code
backgroundcolor=\color{white},  % choose the background color. You must add \usepackage{color}
showspaces=false,               % show spaces adding particular underscores
showstringspaces=false,         % underline spaces within strings
showtabs=false,                 % show tabs within strings adding particular underscores
frame=single,                   % adds a frame around the code
tabsize=2,                      % sets default tabsize to 2 spaces
captionpos=b,                   % sets the caption-position to bottom
breaklines=true,                % sets automatic line breaking
breakatwhitespace=false,        % sets if automatic breaks should only happen at whitespace
title=\lstname,                 % show the filename of files included with \lstinputlisting;
                                % also try caption instead of title
%escapeinside={\%*}{*)},         % if you want to add a comment within your code
%morekeywords={*,...}            % if you want to add more keywords to the set
}

%\usepackage{mdwtab}
\usepackage{graphicx}
\usepackage{hyperref}
\usepackage{color}
\usepackage{fullpage} %use this package to the standard 1 inch margins on all sides
\begin{document}
\title{\huge \textbf{Wayne State University \\
 \vline \\
\Huge CSC 4421 - Summer 2016 \\
 Computer Operating Systems Labs\\
 Lab 5 - Pipes}\\
 \vline
 }
\author{\textbf{Instructor}  \vspace{0.2cm} \\ Saeid Balaneshin-kordan\\
}
%\date{\today}
\maketitle

\begin{center}
\Large Points Possible: 100

\textcolor{red}{Due: June 22, 2016}
\end{center}

\section*{Goals}

The purpose of this lab is to help you learn about creating pipes in C.

\section*{Tasks}

\subsection*{Task 1}
In this task, we want to call two bash commands from our program and print out
all the messages related to the operation of the kernel that are recored today in the kernel ring buffer. The mentioned two commands can be ``dmesg'' and ``sed''.
\begin{enumerate}
\item setup a pipe 
\item fork a child 
\item For the child environment:
\begin{enumerate}
\item redirect \texttt{STDOUT\_FILENO} to the second element of the pipe file descriptor array (\texttt{fd}) by using \texttt{dup2()} function.
\item close the first element of the pipe file descriptors array (\texttt{fd})
\item execute dmesg command: dmesg -T
by using the function \lstinline$execl$
\end{enumerate}
\item For the parent environment:
\begin{enumerate}
\item redirect \texttt{STDOUT\_FILENO} to the first element of the pipe file descriptor \texttt{fd} by using \texttt{dup2()} function.
\item close the second element of the pipe file descriptor \texttt{fd}
\item in a text file explain what the following line does in bash:
\begin{lstlisting}
sed -n -e '/pattern/,$p'  
\end{lstlisting}
\item execute the above sed command 
 with an approperate pattern
by using the function \lstinline$execl$
\item in a text file store the output of your program and submit it.
\end{enumerate}
\end{enumerate}
\end{document}



\end{document} 

